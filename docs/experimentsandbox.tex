% Options for packages loaded elsewhere
\PassOptionsToPackage{unicode}{hyperref}
\PassOptionsToPackage{hyphens}{url}
%
\documentclass[
]{book}
\usepackage{amsmath,amssymb}
\usepackage{lmodern}
\usepackage{iftex}
\ifPDFTeX
  \usepackage[T1]{fontenc}
  \usepackage[utf8]{inputenc}
  \usepackage{textcomp} % provide euro and other symbols
\else % if luatex or xetex
  \usepackage{unicode-math}
  \defaultfontfeatures{Scale=MatchLowercase}
  \defaultfontfeatures[\rmfamily]{Ligatures=TeX,Scale=1}
\fi
% Use upquote if available, for straight quotes in verbatim environments
\IfFileExists{upquote.sty}{\usepackage{upquote}}{}
\IfFileExists{microtype.sty}{% use microtype if available
  \usepackage[]{microtype}
  \UseMicrotypeSet[protrusion]{basicmath} % disable protrusion for tt fonts
}{}
\makeatletter
\@ifundefined{KOMAClassName}{% if non-KOMA class
  \IfFileExists{parskip.sty}{%
    \usepackage{parskip}
  }{% else
    \setlength{\parindent}{0pt}
    \setlength{\parskip}{6pt plus 2pt minus 1pt}}
}{% if KOMA class
  \KOMAoptions{parskip=half}}
\makeatother
\usepackage{xcolor}
\IfFileExists{xurl.sty}{\usepackage{xurl}}{} % add URL line breaks if available
\IfFileExists{bookmark.sty}{\usepackage{bookmark}}{\usepackage{hyperref}}
\hypersetup{
  pdftitle={Experimental design},
  pdfauthor={cjlortie},
  hidelinks,
  pdfcreator={LaTeX via pandoc}}
\urlstyle{same} % disable monospaced font for URLs
\usepackage{longtable,booktabs,array}
\usepackage{calc} % for calculating minipage widths
% Correct order of tables after \paragraph or \subparagraph
\usepackage{etoolbox}
\makeatletter
\patchcmd\longtable{\par}{\if@noskipsec\mbox{}\fi\par}{}{}
\makeatother
% Allow footnotes in longtable head/foot
\IfFileExists{footnotehyper.sty}{\usepackage{footnotehyper}}{\usepackage{footnote}}
\makesavenoteenv{longtable}
\usepackage{graphicx}
\makeatletter
\def\maxwidth{\ifdim\Gin@nat@width>\linewidth\linewidth\else\Gin@nat@width\fi}
\def\maxheight{\ifdim\Gin@nat@height>\textheight\textheight\else\Gin@nat@height\fi}
\makeatother
% Scale images if necessary, so that they will not overflow the page
% margins by default, and it is still possible to overwrite the defaults
% using explicit options in \includegraphics[width, height, ...]{}
\setkeys{Gin}{width=\maxwidth,height=\maxheight,keepaspectratio}
% Set default figure placement to htbp
\makeatletter
\def\fps@figure{htbp}
\makeatother
\setlength{\emergencystretch}{3em} % prevent overfull lines
\providecommand{\tightlist}{%
  \setlength{\itemsep}{0pt}\setlength{\parskip}{0pt}}
\setcounter{secnumdepth}{5}
\usepackage{booktabs}
\ifLuaTeX
  \usepackage{selnolig}  % disable illegal ligatures
\fi
\usepackage[]{natbib}
\bibliographystyle{apalike}

\title{Experimental design}
\author{cjlortie}
\date{}

\begin{document}
\maketitle

{
\setcounter{tocdepth}{1}
\tableofcontents
}
\hypertarget{introduction}{%
\chapter*{Introduction}\label{introduction}}
\addcontentsline{toc}{chapter}{Introduction}

\includegraphics[width=4in,height=\textheight]{./leaf.png}

Experiments shape the human experience. Experiments are a critical component of all living natural systems encompassing evolution to community dynamics. Experiments in science are creative, iterative, \& source critical thinking. We naturally experiment in art, science, and life. Here, we hone these skills through principles and practice. The principles are here, and the practice is in the form a lab manual entitled \href{https://bookdown.org/cj4nature/designcraft4experiments/}{Designcraft for experiments}.

\hypertarget{course-outline}{%
\subsection*{Course outline}\label{course-outline}}
\addcontentsline{toc}{subsection}{Course outline}

If you are electing to engage with this learning opportunity formally, please see the official course outline for specific details. There are two summative assessments to the lecture principles (again also see lab manual for the work associated with that component of the formal course offering if you are doing for credit).

\begin{enumerate}
\def\labelenumi{\arabic{enumi}.}
\tightlist
\item
  Test (on content of the book and critical design thinking for science).\\
\item
  Grant proposal (for experiment and idea you care about).
\end{enumerate}

\hypertarget{learning-outcomes}{%
\subsection*{Learning outcomes}\label{learning-outcomes}}
\addcontentsline{toc}{subsection}{Learning outcomes}

\begin{enumerate}
\def\labelenumi{\arabic{enumi}.}
\tightlist
\item
  Understand the core concepts of experimental design for any natural science experiment.\\
\item
  Understand key terminology, semantics, and experimental design philosophies.\\
\item
  Critically assess experiments.\\
\item
  Provide visual heuristics and workflows for experiments.\\
\item
  Be able to design \& execute an effective experiment.\\
\item
  Be able to clearly write a well-structured manuscript suitable for publication in a journal.\\
\item
  Be able to write a competitive grant proposal appropriate for a Master's application.
\end{enumerate}

\hypertarget{steps-to-design-success}{%
\subsection*{Steps to design success}\label{steps-to-design-success}}
\addcontentsline{toc}{subsection}{Steps to design success}

\hypertarget{module-1.}{%
\subsubsection*{Module 1.}\label{module-1.}}
\addcontentsline{toc}{subsubsection}{Module 1.}

\begin{itemize}
\tightlist
\item
  Read a very accessible book on experimental design.\\
\item
  Take a test to demonstrate mastery of content and creative design for science experiments.
\end{itemize}

\hypertarget{module-2.}{%
\subsubsection*{Module 2.}\label{module-2.}}
\addcontentsline{toc}{subsubsection}{Module 2.}

\begin{itemize}
\tightlist
\item
  Select a science topic that you care deeply about it and do research on this opportunity.\\
\item
  Write a one-page grant proposal appropriate for a graduate-school funding application.
\end{itemize}

\hypertarget{rationale}{%
\subsubsection*{Rationale}\label{rationale}}
\addcontentsline{toc}{subsubsection}{Rationale}

Experiments are a powerful tool to understand, manage, and explore the world around us. This course will provide you with the terminology and concepts you need to be competitive and effective in research and employment. The lectures include exploration of the key terminology and ideas you need to process experiments. You will also practice design experiments in the labs.

Lectures (or independent but facilitated student learning) include three mental processes.

Read. Think. Create.

In the \textbf{first module} (i.e., a total of 6 weeks allocated but please work at your own pace), we read a book together. This component of the lectures provides you with the critical elements, ideas, tools, and terminology you need to design better experiments. The extent that you develop your knowledge and design skills are evaluated using a test, provided in advance, that you complete on your own time. Lectures with decks are provided and they are the principles that emerged, for me, from reading the book.

In the \textbf{second module} (i.e., a total of 4 weeks blocked), you design an experiment for graduate-level research and prepare an NSERC grant proposal (very short, see guidelines). The primary purpose of this component of the lectures is to provide you with the opportunity to generate a novel, useful research proposal on a topic of your choice. Key readings and discussion are provided to support your development and exploration of a topic that further hone your skills.

\hypertarget{schedule}{%
\subsection*{Schedule}\label{schedule}}
\addcontentsline{toc}{subsection}{Schedule}

This is the recommended timing for completing work. Deadlines are firm for submission of summative assessments, but the pacing to get each of those points in time is up to you. In lectures (and labs), we will however work through and discussion the material in this order.

\begin{tabular}{rlll}
\toprule
week & lecture & resource & labs\\
\midrule
1 & intro to course & {}[welcome deck](https://figshare.com/articles/presentation/BIOL3250\_welcome\_deck\_pdf/14944494) & none\\
2 & textbook ch 1 \& 2 & {}[deck\_1](https://figshare.com/articles/presentation/BIOL3250\_ch1/14944506) \& [deck\_2](https://figshare.com/articles/presentation/BIOL3250\_ch2/14944509) & pilot field labs\\
3 & textbook ch 3 \& 4 & {}[deck\_3](https://figshare.com/articles/presentation/BIOL3250\_ch3/14944512) \& [deck\_4](https://figshare.com/articles/presentation/BIOL3250\_ch4/14944515) & pilot field labs\\
4 & textbook ch 5 \& 6 & {}[deck\_5](https://figshare.com/articles/presentation/BIOL3250\_ch5/14944518) \& [deck\_6](https://figshare.com/articles/presentation/BIOL3250\_ch6/14944521) & pilot field labs\\
5 & textbook ch 7 \& 8 & {}[deck\_7](https://figshare.com/articles/presentation/BIOL3250\_ch7/14944524) \& [deck\_8](https://figshare.com/articles/presentation/BIOL3250\_ch8/14944530) & collect data for field experiment you chose\\
\addlinespace
6 & textbook ch 9 \& 10 & {}[deck\_9](https://figshare.com/articles/presentation/BIOL3250\_ch9/14944533) \& [deck\_10](https://figshare.com/articles/presentation/BIOL3250\_ch10/14944536) & **submit data and meta-data for field experiment (see course outline for date)**\\
7 & **test** & see rubric provided here \& due date in course outline & work on lab report\\
8 & {}[ten simple rules for awards](https://journals.plos.org/ploscompbiol/article?id=10.1371/journal.pcbi.1005863) & {}[deck](https://figshare.com/articles/presentation/BIOL3250\_career\_awards\_/14944563) & work on lab report\\
9 & {}[ten simple rules for grants](https://journals.plos.org/ploscompbiol/article?id=10.1371/journal.pcbi.0020012) & {}[deck](https://figshare.com/articles/presentation/BIOL3250\_getting\_grants/14944560) & **submit lab report (see course outline for date) - no lab**\\
10 & shark-tank thinking for grants & {}[engineering shark tank](https://peer.asee.org/the-shark-tank-experience-how-engineering-students-learn-to-become-entrepreneurs) \& [medicine shark tank](https://meridian.allenpress.com/jgme/article/12/3/320/441922/Swimming-With-Sharks-Teaching-Residents-Value) \& [education shark tank](https://journals.lww.com/academicmedicine/fulltext/2017/11000/creating\_an\_\_education\_shark\_tank\_\_to\_encourage.24.aspx) & select and complete a data-design lab [life data deck](https://figshare.com/articles/presentation/BIOL3250\_quantified\_life\_data\_/14944575) \& [MTG deck](https://speakerdeck.com/zulainm/magic-the-gathering)\\
\addlinespace
11 & finalize grant proposal, ensure effective experimental design \& **see official course outline for due date** & {}[NSERC criteria](https://www.nserc-crsng.gc.ca/students-etudiants/pg-cs/cgsm-bescm\_eng.asp) \& review rubric provided in course materials \& [Crafting a Sales Pitch](https://eric.ed.gov/?id=EJ980463) \& [deck](https://figshare.com/articles/presentation/BIOL3250\_sales\_pitch\_for\_getting\_grants/17035907) \& [Developing research Qs through grants](https://www.tandfonline.com/doi/abs/10.1080/036012701753122901) & select \& complete a data-design lab\\
12 & grant thinking \& discussion on best principles for experimental design applications in daily life & {}[grant thinking deck](https://figshare.com/articles/presentation/BIOL3250\_grant\_thinking/17078099) \& [daily life deck](https://figshare.com/articles/presentation/BIOL3250\_experimental\_debrief\_/14944566) & **submit lab report (see course outline for date) - no lab**\\
\bottomrule
\end{tabular}

\hypertarget{citation}{%
\subsection*{Citation}\label{citation}}
\addcontentsline{toc}{subsection}{Citation}

Lortie, CJ (2022): Experiment sandbox. figshare. Book. \url{https://doi.org/10.6084/m9.figshare.20442801.v2}

\hypertarget{license}{%
\subsection*{License}\label{license}}
\addcontentsline{toc}{subsection}{License}

This work is licensed under a Creative Commons Attribution-NonCommercial-ShareAlike 4.0 International License.

\hypertarget{principles}{%
\chapter{Principles}\label{principles}}

\includegraphics[width=3in,height=\textheight]{./design.png}

\hypertarget{learning-outcomes-1}{%
\subsection*{Learning outcomes}\label{learning-outcomes-1}}
\addcontentsline{toc}{subsection}{Learning outcomes}

\begin{enumerate}
\def\labelenumi{\arabic{enumi}.}
\tightlist
\item
  Develop familiarity with key terms and concepts in experimental design theory.
\item
  Summarize key principles of design from a contemporary text on topic.\\
\item
  Successfully demonstrate mastery of design ideas.\\
\item
  Apply the principles to real-world challenges provided in a summative test.
\end{enumerate}

\hypertarget{context}{%
\subsection*{Context}\label{context}}
\addcontentsline{toc}{subsection}{Context}

Experimental design theory is a rich and varied domain of meta-science. It is the how-to or academic practice of science (and many other disciplines too). The concepts, terms, and principles have been discussed at length in texts, peer-reviewed publications, and other resources. In this first primary module, we read a contemporary text entitled \href{https://global.oup.com/academic/product/experimental-design-for-the-life-sciences-9780198717355?cc=us\&lang=en\&}{Experimental Design for the Life Sciences Fourth Edition}.

\hypertarget{steps}{%
\subsection*{Steps}\label{steps}}
\addcontentsline{toc}{subsection}{Steps}

\begin{itemize}
\tightlist
\item
  \textbf{Read the book} (at your own pace or along with the course).\\
\item
  Attend lectures as needed. Typically, two chapters per week are proposed and discussed.\\
\item
  Review the optional slide decks (links provided in schedule) from high-level ideas from each chapter (deck number matches chapter).\\
\item
  Review the test, provided right now, to see what your goals should be.
\end{itemize}

\hypertarget{test}{%
\subsection*{Test}\label{test}}
\addcontentsline{toc}{subsection}{Test}

\hypertarget{instructions}{%
\subsubsection*{Instructions}\label{instructions}}
\addcontentsline{toc}{subsubsection}{Instructions}

\begin{enumerate}
\def\labelenumi{\arabic{enumi}.}
\item
  This is a take-home test. Work on it whenever and wherever you are most comfortable and discuss as needed but please do the writing independently because of the plagiarism check tool associated with turnitin.
\item
  You must submit the test as a \textbf{PDF} to \href{http://www.turnitin.com}{turnitin.com}.\\
\item
  Font must be sized at least 11 point, 1 inch margins, and writing must also adhere to the following guidelines. Concepts: no more than two pages (1 page for each question), Synthesis: no more than 2 pages for the single question.
\item
  A sketch is required for the synthesis question, but it does not count towards page limit. You can draw by hand and scan or use Powerpoint or Keynote to make your visual. It can be included as a third and separate page.
\item
  Please ensure the PDF is entitled \textbf{`Surname\_test.PDF'} and that your name and student ID is at the top of the first page of test.
\end{enumerate}

Happy designing!!

\hypertarget{concepts-select-only-two-qs-from-this-section-5-each}{%
\subsubsection*{Concepts (select ONLY two Qs from this section, 5\% each)}\label{concepts-select-only-two-qs-from-this-section-5-each}}
\addcontentsline{toc}{subsubsection}{Concepts (select ONLY two Qs from this section, 5\% each)}

\begin{enumerate}
\def\labelenumi{\arabic{enumi}.}
\tightlist
\item
  How do decide between two versus many levels for a given factor in designing an experiment? Use figures to show how you decide and provide a brief explanation in text form too. Make a general recommendation on the most likely design to best capture variation across many levels whilst still estimating variation/protecting against drop-outs too.\\
\item
  What is a hypothesis, provide a real world example, and explain the difference between a hypothesis and prediction.\\
\item
  It is sometimes useful to unbalance replicates across groups or levels for ethical or practical reasons. Provide a set of guidelines including a checklist of considerations for researchers. Clearly explain and show when you want to unbalance in favour of controls versus treatments and the converse (so that both textbook and Lortie are correct).\\
\item
  You have been hired to design an experiment. Your employer is keen to do it well, but research costs money. They want to do the best possible experiment but balance that against the cost of paying subjects in the experiment to test their new machine. What are some general tools that you can use to decide on how the extent that you recommended to them to replicate? This is a very real problem as sometimes employers/grants want to spend less on replication and you have to justify why you selected the number you did. Explain the concepts and tools clearly to your employer. Remember, they do not have infinite funds either so you have give them the evidence needed to justify whatever they can afford.\\
\item
  It is easy for scientists to fall into the trap of randomizing everything and assuming this solves all problems. Explain randomization from an experimental design perspective and how you would use it to design the sampling of an experiment with heterogeneity between groups of subjects and in applying the treatments.
\end{enumerate}

\hypertarget{synthesis-select-only-one-q-from-this-section-20}{%
\subsubsection*{Synthesis (select ONLY one Q from this section, 20\%)}\label{synthesis-select-only-one-q-from-this-section-20}}
\addcontentsline{toc}{subsubsection}{Synthesis (select ONLY one Q from this section, 20\%)}

\begin{enumerate}
\def\labelenumi{\arabic{enumi}.}
\item
  You have some lovely fruit trees in your orchard, and it is important to you to protect your crop from pests. Unlike some other growers, you prefer to not trap out the small mammals (squirrels \& rodents) but need to keep the ripening fruit on the tree branches till harvest. The animals can eat the fruit that falls to the ground. You have read that metal flashing wrapped around the trunk prevents animals from climbing up. You have 10,000 (CDN) you were going to spend on advertising but want to spend on doing an experiment this year with flashing so that you can be organic and animal friendly when you advertise next year. You have three major fruit orchards on your farm, each separated by a 1km, and you have noticed animal activity in all three locations. The growing season is 6 months long, beginning in May and ending in October, and at each location you have approximately 300 trees. The metal flashing is sold in rolls 10 ft wide and 100 feet long. Each roll costs 1000 (CDN). You have 3 staff, part-time to help you set up experiment. All three locations have fruit trees that are even aged and approximately the same size at 25 feet tall with a trunk diameter at breast height of on average of 3 ft. Please design experiment to determine if metal flashing wrapped around the trunk will work. Also, determine whether the height of placement and width of flashing needed are important considerations. Write a hypothesis, predictions, and the methods with a sketch to show outline of experiment as you are planning on sharing the find in in a local fruit grower publication for others within the region.
\item
  You are managing a running race. Design a `validation' experiment to demonstrate that the system you have developed minimizes bias and inaccuracies. You have a limited budget for equipment but many volunteers. In your design, also include best design principles for the volunteers too. Develop a checklist you provide to them. Explain the design but also do a very clear schematic to show the validation design to the race officials to get your race Ontario Track and Field Certified. You do not need a hypothesis and prediction for a validation protocol, but you do need the protocol to very clear and decouple bias from inaccuracy. Ensure you also have redundancy and a checking system in place for races in case one system fails because these competitions are very important qualifying races for potentially professional athletes.
\item
  You have decided to do some research contract work after your undergraduate before you make your big career decision. You took an experimental design course and are now confident you can advertise yourself as an expert in design to help out companies. You are also really into the physical training literature and sports. You have secured your first contract with a really innovative gym called P3 that trains many pro-athletes including potential draft recruits for the NBA. As part of the research by the gym, they test really novel training techniques including use of hi-res recordings, electronic muscle stimulation (EMS), and also \href{https://en.wikipedia.org/wiki/Transcranial_direct-current_stimulation}{transcranial direct-current stimulation (tDCS)}. Both are basically forms of electrical stimulation of either the muscles or the brain directly and preliminary but accumulating research has shown both can have positive effects on performance. They have hired you to help them use this really expensive equipment they purchased to train athletes but also collect data in a meaningful way to publish and establish their company (gym but they are branding it as a research clinic). Show off what you know here from experimental design using the terms and ideas of avoiding psuedorep, within-subject measures, PPV, how to decide on replication levels, specificity in testing protocols, etc. Remember however, all althletes are paying clients but return many times for training within a single season. Assume they handle approximately 120 athletes pre-season before draft picks, each athlete can visit up to a dozen times over the course of 6 months, and your goal is to design an experiment that tests whether stimulation works. Ensure you test frequency, contrast direct muscle versus brain stim, and whether you need to both. They train only men 17-21 years but they can vary in height from 6ft upwards. Provide a visual outline of the design as well to show to pitch your design to P3.
\item
  The \href{https://sustainabledevelopment.un.org/sdgs}{UN Sustainable Development Goals} are an inspiration. We need them. Design an experiment that can provide evidence of success after 5 years of hard work by a nation for just one of the 17 goals listed. Other researchers will test the remaining goals so ensure your design has some elements of repeatability and that they can compare their findings to yours.
  List your hypothesis, predictions, and include a description of your pilot experiment. Do this in writing. Clearly state how you are invoking best experimental design principles because your written full grant will be reviewed. You also have to present this to the UN for funding as part of the application process (they want to see if you can also communicate and summarize complexity well). You get only three slides and a few minutes to compete for the funding. Prep three visuals to show during your proposal. I recommend you do one figure showing the hypothesis and prediction clearly in a visual form, one figure showing the design, and one figure showing what you anticipate the outcome of the experiment will generate (i.e.~a data visualization plot).
\item
  VW likely failed to communicate clearly to the public about TDI cars. Someone within the \href{https://en.wikipedia.org/wiki/Volkswagen_emissions_scandal}{company programmed a cheat code} that allows the car to pass aircare tests. However, when driving, it produces 4-40x above acceptable emissions. Design an experiment to determine the frequency that this cheat was applied to their TDI models (3 different models with TDI engines) for cars currently on the road, test if they are set to cheat including the amount they pollute, and the extent that the cheat activates in testing within each model (i.e.~is their sensitivity in the cheat turning on). To be clear, the cheat activates when the cars are put on rollers and only two wheels roll. Develop a hypothesis, predictions, and a very clear design. Include a visual sketch of the experimental design in addition to your description. Anticipate the outcome of your testing by providing a set of data visualization figures from the experiment. VW will likely challenge your findings no matter what you report so ensure you have a robust design you can defend by invoking the `best experimental design' principles from your training.
\end{enumerate}

\hypertarget{rubric}{%
\subsection*{Rubric}\label{rubric}}
\addcontentsline{toc}{subsection}{Rubric}

\begin{tabular}{l|l|l|l}
\hline
section & question & comments & value\\
\hline
concepts & 1 & describe 4 design principles or concepts for question, 1 mark for clarity & 5\\
\hline
concepts & 2 & describe 4 design principles or concepts for question, 1 mark for clarity & 5\\
\hline
synthesis & design & replicates, factors, levels balanced, design embodies effective and best principles & 5\\
\hline
synthesis & viability & likelihood experiment will be successful, hypothesis and predictions align with design, does experiment test what is proposed & 5\\
\hline
synthesis & visual & does the experimental visual serve as clear, effective, educational heuristic that illuminates the design, text to a limited extent to list levels, reps, or explain implementation and salient design elements, can a reader inspect figure and understand experimental design and purpose of work & 5\\
\hline
synthesis & outcome & anticipated outcome specifically stated, implication of work proposed, relationship between data or evidence and outcome or hypothesis linked, did you close the loop for the reader so as to promote a meaningful understanding of why you proposed a specific experiment and how the findings will support or reject the hypothesis through the predictions & 5\\
\hline
test\_total & total of above & one point per percent to value and recognize effort and critical thinking & **30**\\
\hline
\end{tabular}

\hypertarget{grant}{%
\chapter{Grant}\label{grant}}

\includegraphics[width=3in,height=\textheight]{./grant.png}

\hypertarget{learning-outcomes-2}{%
\subsection*{Learning outcomes}\label{learning-outcomes-2}}
\addcontentsline{toc}{subsection}{Learning outcomes}

\begin{enumerate}
\def\labelenumi{\arabic{enumi}.}
\tightlist
\item
  Practice the experimental principles.
\item
  Develop a graduate-level experiment for a grant application.\\
\item
  Appreciate the nuances and challenges of designing an experiment.\\
\item
  Innovate on a topic and advance your knowledge for a subject of interest.
\end{enumerate}

\hypertarget{context-1}{%
\subsection*{Context}\label{context-1}}
\addcontentsline{toc}{subsection}{Context}

This is your chance to practice. Practice makes practice. It is also an opportunity to develop and explore a topic you care deeply about. It provides you with a short grant application appropriate for an funding in Canada for graduate school.

\hypertarget{steps-1}{%
\subsection*{Steps}\label{steps-1}}
\addcontentsline{toc}{subsection}{Steps}

\begin{itemize}
\tightlist
\item
  For this module, you write a grant. Pick a topic or challenge.\\
\item
  Review optional decks and lecture materials provided in schedule.\\
\item
  Discuss ideas with colleagues and peers, discuss with instructor as needed, sketch, think, and be scientifically creative.\\
\item
  Review grant instructions provided here and work towards a successful and timely submission.
\end{itemize}

\hypertarget{bonus-resources}{%
\subsection*{Bonus resources}\label{bonus-resources}}
\addcontentsline{toc}{subsection}{Bonus resources}

Consider using these in addition to the slide decks and links provided in the schedule.

\begin{itemize}
\item
  \href{https://figshare.com/articles/online_resource/BIOL3250_grant_template/13224755}{\textbf{sample grant}} outlined with the formatting set up correctly for submission and for evaluation.
\item
  \href{https://journals.plos.org/ploscompbiol/article?id=10.1371/journal.pcbi.1005863}{Ten simple rules for writing a career development award proposal}
\item
  \href{https://journals.plos.org/ploscompbiol/article?id=10.1371/journal.pcbi.0020012}{Ten Simple Rules for Getting Grants}
\item
  \href{https://journals.plos.org/ploscompbiol/article?id=10.1371/journal.pcbi.1005373}{Ten simple rules for short and swift presentations}
\end{itemize}

\hypertarget{grant-1}{%
\subsection*{Grant}\label{grant-1}}
\addcontentsline{toc}{subsection}{Grant}

\hypertarget{instructions-1}{%
\subsubsection*{Instructions}\label{instructions-1}}
\addcontentsline{toc}{subsubsection}{Instructions}

\begin{enumerate}
\def\labelenumi{\arabic{enumi}.}
\item
  Review the guidelines for graduate funding in Canada. \href{http://www.nserc-crsng.gc.ca/students-etudiants/pg-cs/cgsm-bescm_eng.asp}{Canada Graduate Scholarships-Master's Program} and the \href{http://www.nserc-crsng.gc.ca/ResearchPortal-PortailDeRecherche/Instructions-Instructions/CGS_M-BESC_M_eng.asp}{instructions here if you choose to apply by Dec}.
\item
  Review the rubric provided below.
\item
  Then, select a topic you are interested in, read up on it, do the research on what experiments have been done, innovate, and flex your jedi-design skills.
\item
  Write the proposal and submit as a \textbf{PDF} to \href{http://www.turnitin.com}{turnitin.com}.
\end{enumerate}

\hypertarget{rubric-1}{%
\subsection*{Rubric}\label{rubric-1}}
\addcontentsline{toc}{subsection}{Rubric}

\begin{tabular}{r|l|l|r}
\hline
page & concept & description & value\\
\hline
1 & Title \& summary & Title, your name, contact details, plain language summary < 300 words & 5\\
\hline
2 & Research Proposal & Single page, background, hypothesis, outline of design, methods, significance [see research potential criterion at NSERC for guidance](https://www.nserc-crsng.gc.ca/students-etudiants/pg-cs/cgsm-bescm\_eng.asp) & 10\\
\hline
3 & Lit cited & At least 5 studies published within last 5 years & 5\\
\hline
\end{tabular}

  \bibliography{book.bib,packages.bib}

\end{document}

% Options for packages loaded elsewhere
\PassOptionsToPackage{unicode}{hyperref}
\PassOptionsToPackage{hyphens}{url}
%
\documentclass[
]{book}
\usepackage{amsmath,amssymb}
\usepackage{lmodern}
\usepackage{iftex}
\ifPDFTeX
  \usepackage[T1]{fontenc}
  \usepackage[utf8]{inputenc}
  \usepackage{textcomp} % provide euro and other symbols
\else % if luatex or xetex
  \usepackage{unicode-math}
  \defaultfontfeatures{Scale=MatchLowercase}
  \defaultfontfeatures[\rmfamily]{Ligatures=TeX,Scale=1}
\fi
% Use upquote if available, for straight quotes in verbatim environments
\IfFileExists{upquote.sty}{\usepackage{upquote}}{}
\IfFileExists{microtype.sty}{% use microtype if available
  \usepackage[]{microtype}
  \UseMicrotypeSet[protrusion]{basicmath} % disable protrusion for tt fonts
}{}
\makeatletter
\@ifundefined{KOMAClassName}{% if non-KOMA class
  \IfFileExists{parskip.sty}{%
    \usepackage{parskip}
  }{% else
    \setlength{\parindent}{0pt}
    \setlength{\parskip}{6pt plus 2pt minus 1pt}}
}{% if KOMA class
  \KOMAoptions{parskip=half}}
\makeatother
\usepackage{xcolor}
\IfFileExists{xurl.sty}{\usepackage{xurl}}{} % add URL line breaks if available
\IfFileExists{bookmark.sty}{\usepackage{bookmark}}{\usepackage{hyperref}}
\hypersetup{
  pdftitle={Experimental design},
  pdfauthor={cjlortie},
  hidelinks,
  pdfcreator={LaTeX via pandoc}}
\urlstyle{same} % disable monospaced font for URLs
\usepackage{longtable,booktabs,array}
\usepackage{calc} % for calculating minipage widths
% Correct order of tables after \paragraph or \subparagraph
\usepackage{etoolbox}
\makeatletter
\patchcmd\longtable{\par}{\if@noskipsec\mbox{}\fi\par}{}{}
\makeatother
% Allow footnotes in longtable head/foot
\IfFileExists{footnotehyper.sty}{\usepackage{footnotehyper}}{\usepackage{footnote}}
\makesavenoteenv{longtable}
\usepackage{graphicx}
\makeatletter
\def\maxwidth{\ifdim\Gin@nat@width>\linewidth\linewidth\else\Gin@nat@width\fi}
\def\maxheight{\ifdim\Gin@nat@height>\textheight\textheight\else\Gin@nat@height\fi}
\makeatother
% Scale images if necessary, so that they will not overflow the page
% margins by default, and it is still possible to overwrite the defaults
% using explicit options in \includegraphics[width, height, ...]{}
\setkeys{Gin}{width=\maxwidth,height=\maxheight,keepaspectratio}
% Set default figure placement to htbp
\makeatletter
\def\fps@figure{htbp}
\makeatother
\setlength{\emergencystretch}{3em} % prevent overfull lines
\providecommand{\tightlist}{%
  \setlength{\itemsep}{0pt}\setlength{\parskip}{0pt}}
\setcounter{secnumdepth}{5}
\usepackage{booktabs}
\ifLuaTeX
  \usepackage{selnolig}  % disable illegal ligatures
\fi
\usepackage[]{natbib}
\bibliographystyle{apalike}

\title{Experimental design}
\author{cjlortie}
\date{}

\begin{document}
\maketitle

{
\setcounter{tocdepth}{1}
\tableofcontents
}
\hypertarget{introduction}{%
\chapter*{Introduction}\label{introduction}}
\addcontentsline{toc}{chapter}{Introduction}

\includegraphics[width=4in,height=\textheight]{./leaf.png}

Experiments shape the human experience. Experiments are a critical component of all living natural systems encompassing evolution to community dynamics. Experiments in science are creative, iterative, \& source critical thinking. We naturally experiment in art, science, and life. Here, we hone these skills through principles and practice. The principles are here, and the practice is in the form a lab manual entitled \href{https://bookdown.org/cj4nature/designcraft4experiments/}{Designcraft for experiments}.

\hypertarget{course-outline}{%
\subsection*{Course outline}\label{course-outline}}
\addcontentsline{toc}{subsection}{Course outline}

If you are electing to engage with this learning opportunity formally, please see the official course outline for specific details. There are two summative assessments to the lecture principles (again pls see lab manual for work associated with the component of formal course offering).

\begin{enumerate}
\def\labelenumi{\arabic{enumi}.}
\tightlist
\item
  Test (on content of the book and critical design thinking for science).\\
\item
  Grant proposal (for experiment and idea you care about).
\end{enumerate}

\hypertarget{learning-outcomes}{%
\subsection*{Learning outcomes}\label{learning-outcomes}}
\addcontentsline{toc}{subsection}{Learning outcomes}

\begin{enumerate}
\def\labelenumi{\arabic{enumi}.}
\tightlist
\item
  Understand the core concepts of experimental design for any natural science experiment.\\
\item
  Understand key terminology, semantics, and experimental design philosophies.\\
\item
  Critically assess experiments.\\
\item
  Provide visual heuristics and workflows for experiments.\\
\item
  Be able to design \& execute an effective experiment.\\
\item
  Be able to clearly write a well-structured manuscript suitable for publication in a journal.\\
\item
  Be able to write a competitive grant proposal appropriate for a Master's application.
\end{enumerate}

\hypertarget{steps-to-design-success}{%
\subsection*{Steps to design success}\label{steps-to-design-success}}
\addcontentsline{toc}{subsection}{Steps to design success}

\hypertarget{module-1.}{%
\subsubsection*{Module 1.}\label{module-1.}}
\addcontentsline{toc}{subsubsection}{Module 1.}

\begin{itemize}
\tightlist
\item
  Read a super book on experimental design.\\
\item
  Take a test to demonstrate mastery of content and creative design for science experiments.
\end{itemize}

\hypertarget{module-2.}{%
\subsubsection*{Module 2.}\label{module-2.}}
\addcontentsline{toc}{subsubsection}{Module 2.}

\begin{itemize}
\tightlist
\item
  Select a science topic that you care deeply about it and do research on this opportunity.\\
\item
  Write a one-page grant proposal appropriate for a graduate-school funding application.
\end{itemize}

\hypertarget{rationale}{%
\subsubsection*{Rationale}\label{rationale}}
\addcontentsline{toc}{subsubsection}{Rationale}

Experiments are a powerful tool to understand, manage, and explore the world around us. This course will provide you with the terminology and concepts you need to be competitive and effective in research and employment. The lectures include exploration of the key terminology and ideas you need to process experiments. You will also practice design experiments in the labs.

Lectures/independent but facilitated student learning\\
Read. Think. Create.

In the first module (i.e., a total of 6 weeks allocated but please work at your own pace), we read a book together. This component of the lectures provides you with the critical elements, ideas, tools, and terminology you need to design better experiments. The extent that you develop your knowledge and design skills are evaluated using a test, provided in advance, that you complete on your own time. Lectures with decks are provided and they are the principles that emerged, for me, from reading the book.

In the second module (i.e., a total of 4 weeks blocked), you design an experiment for graduate-level research and prepare an NSERC grant proposal (very short, see guidelines). The primary purpose of this component of the lectures is to provide you with the opportunity to generate a novel, useful research proposal on a topic of your choice. Key readings and discussion are provided to support your development and exploration of a topic that further hone your skills.

\hypertarget{schedule}{%
\subsection*{Schedule}\label{schedule}}
\addcontentsline{toc}{subsection}{Schedule}

This is the recommended timing for completing work. Deadlines are firm for submission of summative assessments, but the pacing to get each of those points in time is up to you. In lectures (and labs), we will however work through and discussion the material in this order.

\begin{tabular}{rlll}
\toprule
week & lecture & resource & labs\\
\midrule
1 & intro to course & {}[welcome deck](https://figshare.com/articles/presentation/BIOL3250\_welcome\_deck\_pdf/14944494) & none\\
2 & textbook ch 1 \& 2 & {}[deck\_1](https://figshare.com/articles/presentation/BIOL3250\_ch1/14944506) \& [deck\_2](https://figshare.com/articles/presentation/BIOL3250\_ch2/14944509) & pilot field labs\\
3 & textbook ch 3 \& 4 & {}[deck\_3](https://figshare.com/articles/presentation/BIOL3250\_ch3/14944512) \& [deck\_4](https://figshare.com/articles/presentation/BIOL3250\_ch4/14944515) & pilot field labs\\
4 & textbook ch 5 \& 6 & {}[deck\_5](https://figshare.com/articles/presentation/BIOL3250\_ch5/14944518) \& [deck\_6](https://figshare.com/articles/presentation/BIOL3250\_ch6/14944521) & pilot field labs\\
5 & textbook ch 7 \& 8 & {}[deck\_7](https://figshare.com/articles/presentation/BIOL3250\_ch7/14944524) \& [deck\_8](https://figshare.com/articles/presentation/BIOL3250\_ch8/14944530) & collect data for field experiment you chose\\
\addlinespace
6 & textbook ch 9 \& 10 & {}[deck\_9](https://figshare.com/articles/presentation/BIOL3250\_ch9/14944533) \& [deck\_10](https://figshare.com/articles/presentation/BIOL3250\_ch10/14944536) & **submit data and meta-data for field experiment (see course outline for date)**\\
7 & **test** & see rubric provided here \& due date in course outline & work on lab report\\
8 & {}[ten simple rules for awards](https://journals.plos.org/ploscompbiol/article?id=10.1371/journal.pcbi.1005863) & {}[deck](https://figshare.com/articles/presentation/BIOL3250\_career\_awards\_/14944563) & work on lab report\\
9 & {}[ten simple rules for grants](https://journals.plos.org/ploscompbiol/article?id=10.1371/journal.pcbi.0020012) & {}[deck](https://figshare.com/articles/presentation/BIOL3250\_getting\_grants/14944560) & **submit lab report (see course outline for date) - no lab**\\
10 & shark-tank thinking for grants & {}[engineering shark tank](https://peer.asee.org/the-shark-tank-experience-how-engineering-students-learn-to-become-entrepreneurs) \& [medicine shark tank](https://meridian.allenpress.com/jgme/article/12/3/320/441922/Swimming-With-Sharks-Teaching-Residents-Value) \& [education shark tank](https://journals.lww.com/academicmedicine/fulltext/2017/11000/creating\_an\_\_education\_shark\_tank\_\_to\_encourage.24.aspx) & select and complete a data-design lab [life data deck](https://figshare.com/articles/presentation/BIOL3250\_quantified\_life\_data\_/14944575) \& [MTG deck](https://speakerdeck.com/zulainm/magic-the-gathering)\\
\addlinespace
11 & finalize grant proposal, ensure effective experimental design \& **see official course outline for due date** & {}[NSERC criteria](https://www.nserc-crsng.gc.ca/students-etudiants/pg-cs/cgsm-bescm\_eng.asp) \& review rubric provided in course materials \& [Crafting a Sales Pitch](https://eric.ed.gov/?id=EJ980463) \& [deck](https://figshare.com/articles/presentation/BIOL3250\_sales\_pitch\_for\_getting\_grants/17035907) \& [Developing research Qs through grants](https://www.tandfonline.com/doi/abs/10.1080/036012701753122901) & select \& complete a data-design lab\\
12 & grant thinking \& discussion on best principles for experimental design applications in daily life & {}[grant thinking deck](https://figshare.com/articles/presentation/BIOL3250\_grant\_thinking/17078099) \& [daily life deck](https://figshare.com/articles/presentation/BIOL3250\_experimental\_debrief\_/14944566) & **submit lab report (see course outline for date) - no lab**\\
\bottomrule
\end{tabular}

\hypertarget{citation}{%
\subsection*{Citation}\label{citation}}
\addcontentsline{toc}{subsection}{Citation}

Lortie, CJ (2022): Biology for environmental management pocketguide. figshare. Book. \url{https://doi.org/10.6084/m9.figshare.15031752.v3}

\hypertarget{license}{%
\subsection*{License}\label{license}}
\addcontentsline{toc}{subsection}{License}

This work is licensed under a Creative Commons Attribution-NonCommercial-ShareAlike 4.0 International License.

\hypertarget{principles}{%
\chapter{Principles}\label{principles}}

\includegraphics[width=3in,height=\textheight]{./design.png}

\hypertarget{learning-outcomes-1}{%
\subsection*{Learning outcomes}\label{learning-outcomes-1}}
\addcontentsline{toc}{subsection}{Learning outcomes}

\begin{enumerate}
\def\labelenumi{\arabic{enumi}.}
\tightlist
\item
  Explore one set of dimensions associated with a changing climate.
\item
  Link science for cities to your life.\\
\item
  Explore one tool that can enable replicable solutions.
\end{enumerate}

\hypertarget{context}{%
\subsection*{Context}\label{context}}
\addcontentsline{toc}{subsection}{Context}

We experience weather but live with climate. Climate is complex. Climate change has both effects on us and is a response to many drivers including anthropenic processes. The reading provided examines urban effects specifically \citep{RN5995}. Use the ten simple rules suggested to structure your analysis of this paper \citep{RN6861}. There is at least one solution that can enable better science. Replication and being able to test the same ideas again can be done using an open-source, and free, programming language to work with data, draw plots, and do statistics. \href{https://www.r-project.org}{R} is one such tool, and it can be used to promote solutions for others to try with their challenges because the code can be shared (i.e.~fondly recall math classes, show your work), and this documentation of science using data improves our solutions \citep{RN4523}.

\hypertarget{test-questions}{%
\subsection*{Test questions}\label{test-questions}}
\addcontentsline{toc}{subsection}{Test questions}

\begin{enumerate}
\def\labelenumi{\arabic{enumi}.}
\tightlist
\item
\end{enumerate}

\hypertarget{rubric}{%
\subsection*{Rubric}\label{rubric}}
\addcontentsline{toc}{subsection}{Rubric}

\begin{enumerate}
\def\labelenumi{\arabic{enumi}.}
\tightlist
\item
\end{enumerate}

\hypertarget{grant}{%
\chapter{Grant}\label{grant}}

\includegraphics[width=3in,height=\textheight]{./grant.png}

\hypertarget{learning-outcomes-2}{%
\subsection*{Learning outcomes}\label{learning-outcomes-2}}
\addcontentsline{toc}{subsection}{Learning outcomes}

\begin{enumerate}
\def\labelenumi{\arabic{enumi}.}
\tightlist
\item
  Explore one set of dimensions associated with a changing climate.
\item
  Link science for cities to your life.\\
\item
  Explore one tool that can enable replicable solutions.
\end{enumerate}

\hypertarget{context-1}{%
\subsection*{Context}\label{context-1}}
\addcontentsline{toc}{subsection}{Context}

We experience weather but live with climate. Climate is complex. Climate change has both effects on us and is a response to many drivers including anthropenic processes. The reading provided examines urban effects specifically \citep{RN5995}. Use the ten simple rules suggested to structure your analysis of this paper \citep{RN6861}. There is at least one solution that can enable better science. Replication and being able to test the same ideas again can be done using an open-source, and free, programming language to work with data, draw plots, and do statistics. \href{https://www.r-project.org}{R} is one such tool, and it can be used to promote solutions for others to try with their challenges because the code can be shared (i.e.~fondly recall math classes, show your work), and this documentation of science using data improves our solutions \citep{RN4523}.

\hypertarget{grant-guidelines}{%
\subsection*{Grant guidelines}\label{grant-guidelines}}
\addcontentsline{toc}{subsection}{Grant guidelines}

\begin{enumerate}
\def\labelenumi{\arabic{enumi}.}
\tightlist
\item
\end{enumerate}

\hypertarget{rubric-1}{%
\subsection*{Rubric}\label{rubric-1}}
\addcontentsline{toc}{subsection}{Rubric}

\begin{enumerate}
\def\labelenumi{\arabic{enumi}.}
\tightlist
\item
\end{enumerate}

  \bibliography{book.bib,packages.bib}

\end{document}
